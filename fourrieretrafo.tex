\section{Fourrierereihe}
\subsection{Erweiterung von der Fourriereihe}
Bis jetzt haben wir nur periodische Funktionen $f(x) = f(x+T)$ in Frequenzen aufgeteilt, also in eine Fourriereihe aufgeteilt. Jetzt stellt sich natürlich die Frage geht dass denn mit jeder beliebigen Funktion $f(x) \neq f(x+T)$?
Der einfachste Ansatz und das was wir hier machen werden ist schon von einer periodischen Funktion auszugehen, allerdings wiederholt sich die Funktion erst in der Unendlichkeit.
Dann können wir unsere Funktion wieder in ihre Bestandteile auflösen:
\begin{equation}
	f(x) = \sum_j \exp (-i (k+\Delta k *j) *x)
\end{equation}
Da
